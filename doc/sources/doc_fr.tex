
\documentclass[a4paper,11pt]{article}

\usepackage[frenchb]{babel}
\usepackage[utf8]{inputenc}
\usepackage[T1]{fontenc}


\title{K-FetOS / ANFK}

\author{Martin \textsc{Vassor}, Rafaël \textsc{Bocquet} \& Théo \textsc{Laurent}}


\begin{document}
\maketitle
\tableofcontents
\section{Présentation}
Ce projet est un micro-noyau pour RaspberryPi. Il a été commencé en juillet 2014
pour le gestionnaire de processus. Il a ensuite été complètement repris en mai 
2015 dans le cadre du projet du cours Systèmes et Réseaux de M. Pouzet et 
M. Bourke à l'ENS.
\section{Dépendances}
Ce projet est un projet pour RaspberryPi et n'a été testé que sur cette 
platforme, néanmoins, il devrait fonctionner sur d'autres plateformes équipées 
d'un processeur ARM1176JZF-S, sous réserve d'adapter les drivers.
Sont requis pour construire le projet : 
\begin{enumerate}
	\item gcc, as et ld pour ARM pour construire le projet
	\item dialog pour l'interface de construction
\end{enumerate}
Si vous souhaitez cross-compiler le projet, vous devez aller modifier les 
Makefiles pour mettre le nom de votre chaine de compilation à la place.

\section{Build}
Pour construire le projet, lancez $\mathtt{./lanch_me.sh}$ à la racine du 
projet. Vous aurez alors une interface qui vous proposera les options les plus
courantes (construire, nettoyer, lire le readme, etc...). En cas de problème de
compilation, n'hésitez pas à envoyer un mail.

\section{Organisation des sources}
\begin{verbatim}
.
+-- doc	:			La documentation
|   +-- sources	:		Les sources de la documentation
+-- kernel :			Les sources du noyau
|   +-- driver :		Les drivers
|   |   +-- drawing :		L'affichage
|   |   +-- frameBuffer :	La gestion du framebuffer, du GPU
|   |   +-- gpio :		Les entrée/sorties GPIO
|   |   +-- mailbox :		Le mailbox
|   |   +-- timer :		L'horloge interne
|   +-- fs :			Le système de fichier
|   +-- mem :			La gestion de mémoire
|   +-- processManager :	La gestion de processus
|   +-- syscall	:		Les appels systèmes
+-- utils :			Les fonctions utilitaires
    +-- maths :			Les fonctions mathématiques
    +-- string :		Le traitement des chaines de caractères
\end{verbatim}

\section{Gestionnaire de processus}
Le gestionnaire créé, transfert, et supprime les processus. Du point de vue du 
noyau, un processus est un état des registres, ainsi qu'un espace en mémoire 
reservé. Il maintient une liste des processus en cours d'execution et une liste 
des processus stoppés.

À la création, chaque processus se voit alloué un espace de 2MiB, dont 1MiB pour
les allocations dynamiques et 1MiB pour la stack. Lors de la création, le 
gestionnaire de processus demande au gestionnaire de mémoire ces 2MiB. Il 
instancie ensuite le processus et le même dans la liste des processus stoppés.

Les transferts de processus se font par appels de la fonction $\mathtt{yield}$.
À ce moment, le gestionnaire sauve l'état des registres, et recharge les 
registres du processus suivant. À noter, la fonction $\mathtt{yield}$ indique 
juste que le processus laisse la main, mais il est tout à fait possible que le
gestionnaire lui redonne tout de suite.

La suppression de processus consiste juste à supprimer la structure du processus
courant, à rendre la mémoire, et à passer au processus suivant.

\section{Gestionnaire de mémoire}
Le gestionnaire de mémoire fonctionne en deux partie : une partie qui alloue 
uniquement des blocks de 2MiB (dans $\mathtt{freeMemory.[ch]}$), et une partie
qui alloue pour chaque processus des blocks de taille variable (dans 
$\mathtt{allocationTable.[ch]}$).

Les blocks de 2MiB sont alloués uniquement sur demande du noyau, typiquement à 
la création de processus. Le gestionnaire utilise un tableau pour stocker 
lesquels sont utilisés et lesquels sont vides (un bit par block). On ne gère que 
les 16 premiers blocks de 2MiB (le tableau fait 2 char).

Les blocks de tailles arbitraires sont alloués par processus, dans la zone de 
2MiB. Le gestionnaire utilise le \emph{Buddy System}. L'arbre des blocks utilsés
est enregistré directement dans la structure du processus maintenue par le 
noyau. À noter qu'une demande d'un bloc de moins de 64B retournera un bloc de 
64B pour des raisons de consommation de mémoire : diviser par deux la taille du 
plus petit bloc allouable multiplie par deux la taille de l'arbre.

\section{Système de fichiers}
Le système de fichier ne contient que des fichiers. Les dossiers sont une 
interpretation des fichiers. 

Un fichier est un structure $\mathtt{fileDescriptor}$ contenant :
\begin{enumerate}
	\item La taille du fichier
	\item Un pointeur vers le descripteur de fichier parent.
	\item Un pointeur vers sa propre référence
	\item Un pointeur vers les données
\end{enumerate}

En outre, les fichiers sont référencés par des structures $\mathtt{fileRef}$ qui
contiennent : 
\begin{enumerate}
	\item La longueur du nom du fichier
	\item Un pointeur vers le nom du fichier
	\item Un pointeur vers le descripteur de fichier
\end{enumerate}

Un dossier est juste un fichier dont les données sont des copies des 
$\mathtt{fileRef}$s.

Pour l'écriture et la lecture de fichier, une copie du $\mathtt{fileDescriptor}$
est faite, dont le pointeur avance au fur et à mesure de l'écriture, et la 
taille diminue en même temps.

\section{Ce qu'il reste à faire}
\subsection{Gestion des exceptions}
Afin de pouvoir traiter les syscalls comme des exceptions, il nous faut un 
mecanisme d'exceptions. Celà nous permettrait d'avoir des programmes qui ne sont
pas compilés en même temps que le noyau.

\subsection{MMU}
Afin de pouvoir finir la gestion des $\mathtt{fork}$, il nous faudrait une MMU. 
Il sera alors facile de forker, juste en copiant les 2MiB alloués au processus, 
et en changeant la référence dans la MMU.

\end{document}

